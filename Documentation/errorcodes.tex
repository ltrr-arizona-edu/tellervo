\chapter{XML Error Codes}

\begin{longtable}{lll}
  \caption{The Corina webservice provides error feedback by means of an error code and description.}\\
  \toprule
  \textbf{Section} & \textbf{Code} & \textbf{Description} \\
  \endhead
  \midrule
  & & \\

  \textbf{General}
  & 001 & Error connecting to database \\
  & 002 & Generic SQL error \\
  
  & & \\
  \midrule
  & & \\
  
  \textbf{Authentication}
  & 101 & Authentication failed \\
  & 102 & Login required \\
  & 103 & Permission denied \\
  & 104 & Unsupported request \\
  & 105 & Invalid server nonce \\
  & 106 & User unknown \\
  & 107 & Unsupported client \\
  & 108 & Unsupported client version \\

  & & \\
  \midrule
  & & \\

  \textbf{Miscellaneous} 
  & 666 & Unknown Error \\
  & 667 & Program bug \\

  & & \\
  \midrule
  & & \\

  \textbf{Internal} 
  & 701 & Internal SQL error \\
  & 702 & Feature not yet implemented \\
  & 703 & Invalid XML being returned by webservice \\
  & 704 & Configuration error \\

  & & \\
  \midrule
  & & \\

  \textbf{User} 
  & 901 & Invalid user parameter(s) \\
  & 902 & Missing user parameter(s) \\
  & 903 & No records match \\
  & 904 & Parameters too short \\
  & 905 & Invalid XML request \\
  & 906 & Record already exists \\
  & 907 & Foreign key violation \\
  & 908 & Unique constraint violation \\
  & 909 & Check constraint violation \\
  & 910 & Invalid data type \\
  & 911 & Series with this version number already exists \\


  \bottomrule
\end{longtable}