
\chapter{The Corina server}
\label{txt:servermaintenance}

For basic day-to-day running of the Corina server, you simply need to make sure that the server is running.  All other interaction and managment (creating users, granting permissions, accessing data) is done through the Corina desktop application.  This section, however, outlines a number of aspects of the server that advanced users may find useful.

\section{Extending the Virtual Appliance}

For those of you that are unfamiliar with Linux, the basic command line prompt is not likely to be very comfortable.  If you are interesting in looking at the server in more detail you may therefore prefer to install a full graphical interface.  Unlike Windows, there are a number of different graphical interfaces (or desktops) to choose from in Linux, the most popular being Gnome and KDE.  To install one of these you need to type one commands listed below.  The first line installs Gnome and the second KDE. Windows users that are new to Linux may find KDE more familiar than Gnome.

 \code{sudo apt-get install ubuntu-desktop}
 \code{sudo apt-get install kubuntu-desktop}

\section{Security}

The basic installation of the Corina server includes the standard configuration for Apache, PHP and PostgreSQL.  Although these products are considered secure by default, there are a number of measures that can be taken to make them more so.  If your server is only accessible within your local intranet (e.g. behind a robust firewall) then you may not feel it necessary to modify the standard setup.  Precautions may be deemed more important if you server is accessible from the internet.  In this case it would be wise to contact your local network administrator for further information.

\subsection{Usernames and passwords}

There are a number of default usernames and passwords setup on your server.  If your server is accessible for the internet we strongly advise you to change these defaults and anyone with knowledge of the Corina server could access and compromise your machine.

\begin{description*}
 \item[System user] - these are the credentials you use to log in to the command prompt in your Corina Virtual Appliance.  By default the user is `corina' and the password is `w3l0v3tr33s'.  To change this log in to the command prompt and type \verb|passwd| and follow the instructions.
 \item[Database user] - these are the credentials used by the webservice to read and write to the database.  
 \item[Corina admin user] - these are the admin credentials that you use to log in with in your Corina desktop application.  Be default the user is `???' and the password is `???'.  To change these open the Corina desktop application, then go to Admin then Change password.
\end{description*}

\subsection{Authentication and encryption}
 
Corina uses a relatively sophisticated method to ensure that unauthorised users cannot access the Corina database through the webservice.  It is loosely based around http digest authentication and uses a challenge and response scheme.  This makes use of cryptographic hashes (a relatively short digital fingerprint of some data but which cannot be decompiled to retrieve the original data) and nonces (a pseudo-random string used just once). All hashes used in the Corina webservice use the MD5 algorithm. This decision will be periodically reviewed to ensure that MD5 is the most appropriate and secure algorithm to use. Whilst an MD5 hash of a short phrase can be compromised, the length and randomness of the original data means with current cracking techniques this is essentially impossible.   For a complete description of Corina's authentication procedure see section \ref{txt:authentication}.

The default Corina server setup, however, uses standard HTTP protocol to communicate between the server and the desktop application.  This is the same protocol used for the majority of web pages on the internet and a determined hacker could eavesdrop on this communication.  Depending on how important and private you perceive your data you may choose to use Secure Socket Layer (SSL) to encrypt this communication.  This is the same technology used by websites such as online banking.  To make full use of this upgrade in security you will however also require a SSL certificate from an official licensing authority.  These certificates typically cost several hundred dollars per year. 


% TODO Describe how to enable SSL

\section{Directly accessing the database}

Although the Corina database is designed to only be accessed by the Corina desktop application via the Corina server's webservice, you may decide that you'd like to directly access the database yourself.  For instance, you may like to write complicated SQL queries to probe your database in ways not currently supported by the Corina desktop client. 

\warn{Any changes made to the database may have drastic consequences.  We strongly recommend that you never write changes directly to the database as this can cause loss of data and corrupt future upgrades to Corina.}

\subsection{PGAdmin3}
One of the easiest ways to access the PostgreSQL database is through the application PGAdmin3.  This is a cross-platform open source application for communicating with PostgreSQL databases.  You can install PGAdmin3 on your desktop computer and access the remotely running database using your database user credentials.  By default, PostgreSQL runs on port 5432.

%% TODO check pg_hba.conf settings and finish this section

\subsection{ODBC}
It is also possible to connect to your Corina database via an ODBC connection.  This allows limited access to the database from a variety of database applications including programs like Microsoft Access for which further details are given here.   To use ODBC you will need to install the PostgreSQL ODBC driver (\url{http://www.postgresql.org/ftp/odbc/}) on your desktop computer.

Once you've installed the driver you can then open a blank database in Access and go to Files, Get external data then Link tables.  In the file dialog box change the file type to ODBC Databases().  Next, select the PostgreSQL Unicode driver, then fill out the server details.  You should then be able to open the tables and views from the Corina server database directly from within Access as if they were local tables.  Be warned though that Access and ODBC have many limitations compared to PostgreSQL, especially with regards data types.  For this reason we \emph{strongly} recommend using this for read only purposes.  Using the ODBC connection to write changes to your PostgreSQL database is quite likely to cause serious issues. 


\subsection{PSQL}
The final, and most advanced method is to use the psql client on your server.  This is a command line client which can be used to interrogate the database.  If you're not already familiar with psql it is unlikely that this is a good method for you to use!