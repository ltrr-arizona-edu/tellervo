
\chapter{Installation}

Corina is made up of two packages; the Corina desktop application and the Corina database server.  Corina was designed primarily for laboratories with multiple users, each running the Corina desktop application on their own computer connecting to a single central server containing the lab's data.  In this situation the Corina server would be run on a separate computer to those running the desktop client, but this need not necessarily be the case.  It is perfectly possible to run both the server and the client on the same computer.  This is likely to be the situation if you simply want to try out Corina, if you don't have a separate server, or if you do not work in a multi-user laboratory.


\section{Desktop application}

Installation packages for the Corina desktop application are available for Windows, MacOSX and Ubuntu Linux.  Corina can also be run on other operating systems as long as they support Java 6.

To install Corina, download the installation file for your operating system from \url{http://dendro.cornell.edu/corina/download.php}. The website should provide you with a link to the installer for your current operating system:

\begin{description}
\item \includegraphics[width=3mm]{Images/windows.png} \textbf{Windows} -- Run the setup.exe and follow the instructions. If you do not have Java installed the installer will direct you to the Java website where you can get the latest version. Once installed, Corina can be launched via the Start menu.

\item \includegraphics[width=3mm]{Images/mac.png} \textbf{Mac OS X} -- As mentioned above, Corina requires Java 6. Although MacOSX ships with Java installed, unfortunately Apple have been very slow to provide Java 6. Although it was released in 2006, it was not until August 2009 that Apple made Java 6 available as part of v10.6 (snow leopard). Corina can therefore only be run on Snow Leopard or later. To do so, download the dmg disk image file and mount it by double clicking on it. Drag the Corina.app into your applications folder and copy the manual and license files to somewhere convenient in your documents folder.  For the more adventurous there is the possibility that you could run Corina using SoyLatte instead of the standard Java installation that comes with the operating system.  This could be a possible method for running Corina even on earlier versions of MacOSX but is unsupported and largely untested.

\item \includegraphics[width=3mm]{Images/ubuntu.png} \textbf{Ubuntu Linux} --  A deb file is available which was designed for use on Ubuntu distributions but should work on any Debian based system. Install using your favorite package management system e.g. \verb|sudo dpkg --install corina\_2.xx-1\_all.deb|. On Ubuntu and similar distributions, the package should add a Corina shortcut to your applications menu. Alternatively you can start Corina from the command line by typing corina.

\item \includegraphics[width=3mm]{Images/java.png} \textbf{Other operating systems} -- Make sure you have Java 6 installed, then download the Corina jar file to your hard disk. You can run Corina from the command line by typing: \verb|java -jar corina.jar|.
\end{description}

\subsection{Mapping support}

Corina includes 3D mapping for visualization of sampling locations. Although this is not necessary for most tasks, to make use of the mapping functions you will require a OpenGL 3D capable graphics card. To check whether your computer already supports 3D mapping, open Corina, go to Admin, then Site map.  Corina will warn you if your graphics card is not supported.

All MacOSX computers should automatically support OpenGL.  Most Windows and Linux computers made since 2006 should also support OpenGL, however, this does require proper drivers to be installed. In some cases Windows computers may include a compatible graphics card, but may only have the default Windows video drivers installed.  If you are having trouble with the mapping in Corina make sure you have installed the most recent drivers for your graphics card.  Linux users may be required to install proprietary graphics drivers.  

The mapping component of Corina makes use of NASA's open source World Wind Java.  NASA's website \url{http://worldwind.arc.nasa.gov/} contains further information and instructions that you may find helpful if you are having problems getting the mapping to work.  

\section{Server installation}

For the Corina desktop application to be useful you will also require access to a Corina server.  If you are running Corina in a lab where the Corina server has already been set up by your systems administrator, you can skip this section.

The Corina server is made up of a number of components, which unlike the desktop client, can not be easily combined together into cross-platform packages.  Although all the constituent components are open-source and available for all major platforms, building and maintaining separate packages for each platform is too large a task for a small development team.  To conserve resources, we therefore made the decision to utilize Virtual Machine technology to ensure that the Corina server could still be run on all major operating systems.  This means that we can package the Corina server for a single operating system (Ubuntu Linux) and then distribute it as a Virtual Appliance that can be run as a program on your normal operating system. 

The Corina server is therefore available via two main methods.  The first is as a VirtualBox Virtual Appliance which can be run on any major operating system, the second is as an Ubuntu package for running natively on a Linux server.  The source code for the server is also available so it is perfectly possible for more experienced users to set up the Corina server to run natively on other platforms.  For this you will require knowledge of Apache 2, PHP and PostgreSQL.

\subsection{Virtual Appliance - all operating systems}

To run the Corina server Virtual Appliance, you will first need to download and install VirtualBox from \url{http://www.virtualbox.org}.  Installation packages are available for Windows, MacOSX, OpenSolaris and many Linux distributions.

Once you have VirtualBox installed, you will then need to download the Corina server from the Cornell website \url{http://dendro.cornell.edu/corina}.  This package contains a bare-bones Ubuntu Linux server with everything required to run the Corina server installed and ready to use.  As VirtualBox, the entire Ubuntu operating system and Corina server components are all open source there are no license fees to pay.

Open VirtualBox and go to File, Import Appliance, then follow the wizard selecting the Corina server appliance file when prompted.  Once installed you can run your server by highlighting it in the list and pressing start.  The server will boot up in a window alongside your normal operating system and eventually reach a login prompt.  To save on download size and disk space only the essential packages to make the server run have been installed.  This means there is no graphical interface just a command line.  Hopefully this should not be a problem as once set up, the only interaction needed with the Virtual Appliance will be through the normal Corina desktop application.  If you would prefer to use a graphical interface to the server this can be easily installed.  See chapter \ref{txt:servermaintenance} for further details.latex define environment

Before you can use your server you will need to know the IP address that the server has been assigned by your network.  To do this login at the prompt with the default admin credentials: user -- corina; password -- w3l0v3tr33s.  Once logged in, type \verb|corina --test| and a basic configuration test will be performed.  If all is well, all tests will be passed and it will tell you the URL of your new server.  You will need to set your Corina client to point at this webservice to use your server.



\subsection{Ubuntu native installation}

If you are fortunate enough to be running Ubuntu then the native Ubuntu deb package is the best and easiest method for installing the Corina server, otherwise see section \ref{txt:virtualAppliance} to install the server as a Virtual Appliance.  

To install the Corina server in Ubuntu simply download the deb package from the Cornell server \url{http://dendro.cornell.edu/corina} and install with your favourite package manager.  For instance, to install from the command line simply type \verb|sudo dpkg --install corina-server.deb|. The package will automatically run a configuration script to assist with creating a database user, building the Corina PostgreSQL database, setting database permissions and setting up the Apache webservice.  The configuration ends with a test routine to check all services are set up correctly and if so, will provide you with the URL of the newly configured Corina webservice.

\subsection{Advanced install on other operating systems}

As mentioned previously, the limited resources available for Corina development means that we have been unable to produce native installers for platforms other that Ubuntu.  If you are an experience systems administrator though, it should not be too difficult to set up the Corina server manually.  

The Corina server is essentially a PostgreSQL database accessed via a PHP webservice running on Apache 2.  The following dependencies are therefore required: postgresql-8.4; postgis; postgresql-contrib-8.4; postgresql-8.4-pljava; sun-java6-jre; apache2; php5; php5-pgsql; php5-curl; php5-mhash.

The basic procedure for installation is as follows:

\begin{itemize*}
 \item Install all dependencies
 \item Create PostgreSQL database from Corina template SQL file
 \item Set up a database user and provide access to the server in the pg\_hba.conf file
 \item Give this user read and write permissions to the database
 \item Copy the webservice code into a web accessible folder
 \item Set up Apache to see this folder by creating an entry in the sites-enabled folder
 \item Restart PostgreSQL and Apache and check you can access the webservice from a web browser
\end{itemize*}

